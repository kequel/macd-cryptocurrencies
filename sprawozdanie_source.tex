\documentclass[a4paper,12pt]{article}
\usepackage[T1]{fontenc}
\usepackage[utf8]{inputenc}
\usepackage[polish]{babel}
\usepackage{amsmath}
\usepackage{graphicx}
\usepackage{float}
\usepackage{caption}
\usepackage{hyperref}

\title{Analiza wskaźnika MACD}
\author{Karolina Glaza \\ \href{https://github.com/kequel}{github.com/kequel}}
\date{Marzec 2025}

\begin{document}

\maketitle

\section{Wstęp}
Wskaźnik MACD, czyli wskaźnik zbieżności/rozbieżności średnich kroczących (ang. moving average 
convergence/divergence), jest jednym z najpopularniejszych narzędzi w analizie technicznej 
instrumentów finansowych. Umożliwia identyfikację sygnałów kupna i sprzedaży, które mogą wspomóc 
ocenę sytuacji rynkowej, lecz należy traktować je jako pomocniczy element analizy technicznej, a nie 
jedyny fundament decyzji inwestycyjnych. 
\\ \\ Składa się z dwóch elementów: 
\begin{itemize}
    \item Linii MACD – różnicy między 12-okresową a 26-okresową wykładniczą średnią kroczącą (EMA),
    \item Linii sygnałowej (SIGNAL) – 9-okresowej EMA obliczonej z wartości MACD.
\end{itemize}
Sygnały transakcyjne generowane są w momencie przecięcia się tych linii. \\ \\
{\centering W projekcie weryfikowano skuteczność MACD dla kryptowaluty Dogecoin w okresie 03/20/2021 - 
03/20/2025.}

\section{Implementacja}
\subsection{Dane i narzędzia}
\subsubsection{Dane:} historyczne ceny zamknięcia Dogecoin (5 lat) z pliku CSV.

\subsubsection{Narzędzia:}
\begin{itemize}
    \item \texttt{pandas} – wczytywanie danych,
    \item \texttt{numpy} – obliczenia EMA,
    \item \texttt{matplotlib} – wizualizacja. \\
\end{itemize}

\subsection{Algorytm}
Do wyznaczenia wskaźnika MACD wykorzystano kroki oparte na obliczaniu wykładniczych średnich kroczących (EMA), które nadają większą wagę nowszym obserwacjom.

\begin{enumerate}
    \item \textbf{Obliczenie EMA:} \\
    \[
    EMA_t = \alpha \cdot x_t + (1 - \alpha) \cdot EMA_{t-1},
    \]
    gdzie $x_t$ to cena zamknięcia w dniu $t$, a $\alpha$ to współczynnik wygładzania, obliczany jako:
    \[
    \alpha = \frac{2}{N + 1},
    \]
    przy czym $N$ oznacza liczbę okresów.
    \item \textbf{Wyznaczenie MACD:} \\
    Różnica między 12-okresową EMA a 26-okresową EMA:
    \[
    MACD_t = EMA_{12,t} - EMA_{26,t}.
    \]
    \item \textbf{Wyznaczenie linii sygnałowej (SIGNAL):} \\
    9-okresowa EMA wyliczona na podstawie wartości MACD:
    \[
    SIGNAL_t = EMA_{9}(MACD).
    \]
    \item \textbf{Generowanie sygnałów transakcyjnych}
    na podstawie przecięć linii MACD i SIGNAL
    \begin{itemize}
        \item Sygnał \textbf{kupna} – gdy MACD przecina SIGNAL od dołu,
        \item Sygnał \textbf{sprzedaży} – gdy MACD przecina SIGNAL od góry.
    \end{itemize}
   Sygnały generowane przez porównanie MACD i SIGNAL dzień po dniu.
\end{enumerate}

\section{Wyniki i analiza}
\subsection{Wykresy [20.03.2021 – 20.03.2025]}

\begin{figure}[H]
  \centering
  \includegraphics[width=\textwidth]{sygnaly_dogecoin.png}
  \par\vspace{0.5em}
  \parbox{0.9\textwidth}{\centering\small Wskaźnik generował liczne sygnały, często w momentach wysokiej zmienności, co prowadziło do strat} \\
\end{figure}

\begin{figure}[H]
  \centering
  \includegraphics[width=\textwidth]{ms_dogecoin.png}
  \par\vspace{0.5em}
  \parbox{0.9\textwidth}{\centering\small Linie MACD i SIGNAL często przecinały się, generując 62 sygnały kupna i 62 sprzedaży. }
\end{figure}

\subsection{Wykresy [10.12.2024 – 20.03.2025]}

\begin{figure}[H]
  \centering
  \includegraphics[width=\textwidth]{sygnaly_dogecoin_100dni.png}
  \par\vspace{0.5em}
  \parbox{0.9\textwidth}{\centering\small Wykres przedstawia zmienność cen Dogecoin w okresie 100 dni, ukazując liczne wahania oraz momenty generacji sygnałów kupna i sprzedaży. }
\end{figure}

\begin{figure}[H]
  \centering
  \includegraphics[width=\textwidth]{ms_dogecoin_100dni.png}
  \par\vspace{0.5em}
  \parbox{0.9\textwidth}{\centering\small Przecięcia ilustrują dynamiczną reakcję wskaźnika na krótkoterminowe zmiany cen.}
\end{figure}


\subsection{Analiza transakcji (ostatnie 100 dni)}

\begin{table}[H]
\centering
\begin{minipage}{0.48\textwidth}
\centering
\caption{Sygnały kupna}
\begin{tabular}{|c|c|}
\hline
\textbf{Data} & \textbf{Cena (USD)} \\
\hline
2025-03-08 & 0.1923 \\
2025-02-01 & 0.3080 \\
2025-01-03 & 0.3797 \\
2024-12-18 & 0.3586 \\
\hline
\end{tabular}
\end{minipage}
\hfill
\begin{minipage}{0.48\textwidth}
\centering
\caption{Sygnały sprzedaży}
\begin{tabular}{|c|c|}
\hline
\textbf{Data} & \textbf{Cena (USD)} \\
\hline
2025-03-13 & 0.1652 \\
2025-02-11 & 0.2531 \\
2025-01-14 & 0.3558 \\
2025-01-02 & 0.3388 \\
\hline
\end{tabular}
\end{minipage}
\end{table}

\paragraph{Transakcja 1:\\}  
Kupno 2024-12-18 (0.3586 USD), Sprzedaż 2025-01-02 (0.3388 USD)\\  
\textbf{STRATA} w wysokości $0.3388 - 0.3586 = -0.0198$ USD \\  
Pomimo pozytywnego sygnału MACD, cena po krótkim wzroście spadła. Wskazuje to na opóźnioną reakcję MACD.

\paragraph{Transakcja 2:\\}  
Kupno 2025-01-03 (0.3797 USD), Sprzedaż 2025-01-14 (0.3558 USD)\\  
\textbf{STRATA} w wysokości $0.3558 - 0.3797 = -0.0239$ USD \\  
MACD wygenerował sygnał kupna po wcześniejszym sygnale sprzedaży. Cena wzrosła tylko chwilowo, po czym nastąpił spadek. Nietrafiony moment wejścia na rynek.

\paragraph{Transakcja 3:\\}  
Kupno 2025-02-01 (0.3080 USD), Sprzedaż 2025-02-11 (0.2531 USD)\\  
\textbf{STRATA} w wysokości $0.2531 - 0.3080 = -0.0549$ USD \\  
Wskaźnik MACD zasugerował kupno, ale spadek cen Dogecoina rozpoczął się niedługo później. Sygnał błędny.

\paragraph{Transakcja 4:\\}  
Kupno 2025-03-08 (0.1923 USD), Sprzedaż 2025-03-13 (0.1652 USD)\\  
\textbf{STRATA} w wysokości $0.1652 - 0.1923 = -0.0271$ USD \\  
Cena nie utrzymała się, Wskaźnik nie przewidział krótkoterminowej korekty.

\paragraph{Wnioski:}
Wszystkie cztery transakcje zakończyły się stratami, mimo że decyzje były podejmowane zgodnie z klasycznymi sygnałami MACD. Wskaźnik ten generował sygnały zbyt późno lub w momentach dużej zmienności, co czyni go \textbf{nieskutecznym narzędziem w analizie krótkoterminowej dla Dogecoin}. \\

\section{Symulacja inwestycyjna}
\subsection{Założenia}
\begin{itemize}
    \item Kapitał początkowy: 1000 USD,
    \item Strategia:  Kupno/sprzedaż całego portfela na każdy sygnał.
\end{itemize}

\subsection{Wykres wartości portfela}
\begin{figure}[H]
  \centering
  \includegraphics[width=\textwidth]{portfel_dogecoin.png}
  \par\vspace{0.5em}
  \parbox{0.9\textwidth}{\centering\small Strategia oparta wyłącznie na MACD doprowadziła do straty 84.6\% kapitału, głównie z powodu częstych fałszywych sygnałów w warunkach wysokiej zmienności Dogecoin.}
\end{figure}

\subsection{Wyniki}
\begin{table}[H]
\centering
\begin{tabular}{|p{6cm}|p{6cm}|}
\hline
\textbf{Wskaźnik} & \textbf{Wartość} \\
\hline
Końcowy kapitał & 155{,}39 USD \\
\hline
Liczba transakcji & 62 (17 zyskownych, 45 stratnych) \\
\hline
Średni zysk na transakcję & +3{,}48\% \\
\hline
Maksymalny zysk & +222{,}44\% \\
\hline
Maksymalna strata & –33{,}25\% \\
\hline
\end{tabular}
\end{table}

\section{Wnioski dla Dogecoin}
\subsection{Ograniczenia MACD}
\begin{itemize}
    \item Opóźnienie sygnałów - MACD reaguje z opóźnieniem na nagłe zmiany cen, szczególnie na rynkach o wysokiej zmienności (jak kryptowaluty).
    \item Fałszywe sygnały - 73\% transakcji zakończyło się stratą z powodu częstych przecięć linii MACD i SIGNAL w krótkich okresach.
\end{itemize}

\subsection{Mocne strony}
\begin{itemize}
    \item Prostota - Łatwa implementacja i interpretacja.
    \item Skuteczność w trendach - W pojedynczych przypadkach (np. maksymalny zysk +222\%) MACD poprawnie identyfikował długoterminowe trendy.
\end{itemize}

\subsection{Możliwe udoskonalenia (dodanie filtrów):}
\begin{itemize}
    \item Ignorowanie małych zmian (ignoruj sygnały, jeśli zmiana ceny <X\%),
    \item Dodanie innych wskaźników i wymaganie potwierdzenie przez nie,
    \item Strategia „kup i trzymaj”:\\ Dla Dogecoin w badanym okresie pasywna inwestycja byłaby bardziej opłacalna. \\\\
\end{itemize}

\section{Porównanie z innymi danymi: Historyczne ceny zamknięcia Bitcoin}
Bitcoin został włączony do analizy,
aby porównać skuteczność wskaźnika MACD na rynku o innej charakterystyce. Jego niższa zmienność i
wyższa kapitalizacja umożliwiają ocenę metod w szerszym kontekście.

\subsection{Wykresy [20.03.2021 – 20.03.2025]}
\begin{figure}[H]
  \centering
  \includegraphics[width=\textwidth]{sygnaly_bitcoin.png}
  \par\vspace{0.5em}
  \parbox{0.9\textwidth}{\centering\small W porównaniu do Dogecoin, ceny Bitcoin wykazują mniejszą zmienność, co przekłada się na rzadsze, lecz bardziej stabilne sygnały. }
\end{figure}

\begin{figure}[H]
  \centering
  \includegraphics[width=\textwidth]{ms_bitcoin.png}
  \par\vspace{0.5em}
  \parbox{0.9\textwidth}{\centering\small Linie MACD i SIGNAL przecinały się, generując 54 sygnały kupna i 55 sprzedaży }
\end{figure}

\subsection{Wykresy [10.12.2024 – 20.03.2025]}

\begin{figure}[H]
  \centering
  \includegraphics[width=\textwidth]{sygnaly_bitcoin_100dni.png}
  \par\vspace{0.5em}
  \parbox{0.9\textwidth}{\centering\small Pomimo mniejszej zmienności niż Dogecoin, pojawiają się momenty generowania sygnałów, które mogą sugerować zarówno początki trendów, jak i często fałszywe alarmy. }
\end{figure}

\begin{figure}[H]
  \centering
  \includegraphics[width=\textwidth]{ms_bitcoin_100dni.png}
  \par\vspace{0.5em}
  \parbox{0.9\textwidth}{\centering\small Mniejsza zmienność nie eliminuje ryzyka fałszywych sygnałów. }
\end{figure}

\subsection{Symulacja inwestycyjna dla innych danych}
\begin{table}[H]
\centering
\begin{tabular}{|p{6cm}|p{6cm}|}
\hline
\textbf{Wskaźnik} & \textbf{Wartość} \\
\hline
Końcowy kapitał & 754{,}02 USD \\
\hline
Liczba transakcji & 54 (18 zysków, 36 strat) \\
\hline
Średni zysk na transakcję & +0{,}94\% \\
\hline
Maksymalny zysk & +43{,}52\% \\
\hline
Maksymalna strata & –14{,}83\% \\
\hline
\end{tabular}
\end{table}


\subsection{Wykres wartości portfela}
\begin{figure}[H]
  \centering
  \includegraphics[width=\textwidth]{portfel_bitcoin.png}
  \par\vspace{0.5em}
  \parbox{0.9\textwidth}{\centering\small W przypadku Bitcoin, przy mniejszej liczby transakcji, portfel wykazuje mniejszy spadek kapitału niż w analizie Dogecoin, co wynika z mniejszej zmienności rynku, choć ogólne rezultaty nadal są niezadowalające.}
\end{figure}
\paragraph{}

\section{Wyniki Dogecoin vs Bitcoin}
Analiza wykazuje, że wskaźnik MACD, mimo że generuje mniej sygnałów na rynku Bitcoin niż na
Dogecoin, również nie gwarantuje sukcesu inwestycyjnego. Mniejsze ryzyko fałszywych sygnałów nie
przekłada się na jednoznaczną przewagę, co sugeruje konieczność łączenia MACD z dodatkowymi
narzędziami analizy technicznej w celu poprawy skuteczności strategii.

\section{Podsumowanie}
Wskaźnik MACD okazał się nieskutecznym narzędziem, gdy stosowany był samodzielnie, zarówno dla
bardzo zmiennych kryptowalut, takich jak Dogecoin, jak i dla bardziej stabilnych, jak Bitcoin. \\\\
W przypadku Dogecoin wysoka zmienność skutkowała wieloma fałszywymi sygnałami, podczas gdy dla
Bitcoin sygnały były rzadsze, ale nadal nie przynosiły oczekiwanej przewagi inwestycyjnej.  \\\\
Analizy wskazują, że MACD pomocny może być jedynie wtedy, gdy jest on wykorzystywany w połączeniu z
innymi narzędziami analizy technicznej oraz odpowiednio dostosowanymi filtrami, co pozwala na lepsze
odzwierciedlenie specyfiki poszczególnych rynków.

\end{document}
